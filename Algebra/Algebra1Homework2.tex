\documentclass[12pt]{report}
\usepackage[margin=0.5in]{geometry}
\usepackage{amssymb,textcomp}
\title{\textbf{Algebra I: Homework 2}}
\author{Lee Fisher}
\date{}
\begin{document}

\textbf{Algebra 1 Homework 2}\\
\indent \textbf{Lee Fisher}\\
\indent \textbf{2017-09-02}

\begin{itemize}
\item Page 22. \#21: Let $G$ be a finite group and $x$ be an element of order $n$. Prove that if $n$ is odd, then
      $x = x^(2k)$ for some $k$.

      Proof. If $n$ is odd, then $n = 2m+1$ for some non-negative integer $m$. Since $x$ is of order $n$, this means
      $x^{2m+1} = 1$. We can multiply both sides by $x$ on the left and arrive at $x^{2m+2} = x$, then by associativity
      $x^{2(m+1)} = 1$. We can set $k = m+1$ and arrive at our conclusion. If $x$ has odd order, then there will exist a
      $k$ for which $x^{2k} = x$.
\
\item Page 22. \#25: Prove that if $x^2 = 1$ for all elements in $G$, then $G$ is abelian.

      Proof. Consider two elements in $G$, $a$ and $b$. We know $a^2 = 1$ and $b^2 = 1$. So consider two elements, $a$
      and $b$, and the element $(ab)^2$. We have $(ab)^2 = 1$ by definition, but also $(ab)^2 = abab = (ab)(ab)$. This
      means $(ab)^{-1} = ab$. We also have that $(ab)^{-1} = b^{-1}a^{-1}$, and since anything squared in $G$ is the
      identity we know $a^{-1} = a$ and $b^{-1} = b$. Therefore $(ab)^{-1} = ba$ but also $(ab)^{-1} = ab$. Finally we
      have $ab = ba$ by transitivity.

\item Page 22. \#35: If $x$ is an element of finite order $n$ in $G$ use the Division Algorithm to show that any integral
      power of $x$ equals one of the elements in the set $\{1,x,x^2, \dots x^{n-1}\}$.

      Proof. Let $m$ be a number greater than $n$. By the division algorithm, $m = qn + r$ where $0 \leq r < n$. Then
      we have: $x^m = x^{qn + r} = (x^n)^q(x)^r = 1^qx^r = x^r$. Since $r < n$ we have $x^r \in \{1,x,x^2,
      \dots x^{n-1}\}$.

\item Page 33. \#6: Write down the cycle decomposition of every element in $S_4$.

      Okay: (1), (12), (13), (14), (23), (24), (34), (12)(34), (13)(24), (14)(23), (123), (124), (132), (134), (142),
      (143), (234), (243), (1234), (1432), (1324), (1423), (1342), (1243).

\item Page 33. \#9: Before doing these problems I'll prove a lemma. The order of an m-cycle is m. Let
$\sigma = (a_1, a_2, a_3, \dots a_m)$. Then for any $i$, $1 \leq i \leq m$, we have
$\sigma(a_i) = a_{(i+1) \textrm{mod m}}$. Then extending we would have $\sigma^k(a_i) = a_{(i+k) \textrm{mod m}}$. This
tells us that $|\sigma| = m$, because $\sigma^m(a_i) = a_{(i+m) \textrm{mod m}} = a_i$ and if $k < m$ then
$(i+k) \textrm{mod m} \neq i$.

      \begin{itemize}
      \item Let $\sigma = (1,2,3,4,5,6,7,8,9,10,11,12)$. For which positive integers $i$ is $\sigma^i$ also a 12-cycle?
            Since $\sigma$ has order 12 we can consider the cyclic subgroup generated by $\sigma$. This subgroup is
            isomorphic to $(\mathbb{Z}_12, +)$ with isomorphism $\phi: \sigma^k \to k \textrm{mod 12}$. The question of
            how many powers of $\sigma$ are 12-cycles is (by the lemma) the same as the question of how many generators
            there are of $\mathbb{Z}_12$. The numbers 1, 5, 7, and 11 will generate $\mathbb{Z}_12$. So, $\sigma^1$,
            $\sigma^5$, $\sigma^7$, and $\sigma^11$ will be 12-cycles.

      \item Let $\tau = (1,2,3,4,5,6,7,8)$. For which positive integers $i$ is $\tau^i$ also an 8-cycle?
            For the same reasons as the last question $\sigma^1$, $\sigma^3$, $\sigma^5$, and $\sigma^7$ will all be
            8-cycles.

      \item Let $\omega = (1,2,3,4,5,6,7,8,9,10,11,12,13,14)$. For which positive integers $i$ is $\omega^i$ also a
            14-cycle? Again for the same reasons, $\sigma^1$, $\sigma^3$, $\sigma^5$, $\sigma^9$, $\sigma^11$, and
            $\sigma^13$ will all by 14-cycles.

      \end{itemize}

\item Page 33. \#13: Show that an element has order $2$ in $S_n$ if and only if it's cycle decomposition is a
      product of commuting two cycles.

      Proof:
      $\rightarrow$ Suppose $\sigma$ is an element of order $2$. We know that $\sigma$ is a product of commuting cycles.
      The order of each cycle is its length. This means that the order of $\sigma$ is the least common multiple of the
      lengths of the commuting cycles that make up $\sigma$. Therefore, if the order of $\sigma$ is 2 then the least
      common multiple of the cycles that make up $\sigma$ must also be 2. This means $\sigma$ is made up of only
      2-cycles.

      $\leftarrow$ Suppose $\sigma$ is product of commuting 2-cycles. Then $\sigma^2$ is a product of commuting 2-cycles
      as well. In $\sigma^2$ every 2-cycle is repeated twice. Since the order of an m-cycle is m, and since the 2-cycles
      commute all the pairs of 2-cycles will cancel and we have the order of $|\sigma| = 2$.

\item Page 36. \#2: Write out the group tables for $S_3$, $D_8$ and $Q_8$.

\begin{table}[h]
\centering
\caption{$S_3$}
\label{Symmetric}
\begin{tabular}{c|cccccc}
$\circ$ & (1)   & (12)  & (13)  & (23)  & (123) & (132) \\
\hline
(1)     & (1)   & (12)  & (13)  & (23)  & (123) & (132) \\
(12)    & (12)  & (1)   & (132) & (123) & (23)  & (13)  \\
(13)    & (13)  & (123) & (1)   & (132) & (12)  & (23)  \\
(23)    & (23)  & (132) & (123) & (1)   & (13)  & (12)  \\
(123)   & (123) & (13)  & (23)  & (12)  & (132) & (1)   \\
(132)   & (132) & (23)  & (12)  & (13)  & (1)   & (123)
\end{tabular}
\end{table}

\begin{table}[h]
  \centering
\caption{$D_8$}
\label{Dihedral}
\begin{tabular}{c|cccccccc}
$\circ$  &$R_0$    &$R_{90}$ &$R_{180}$&$R_{270}$&V        &H        &D        &D'  \\
\hline
$R_0$    &$R_0$    &$R_{90}$ &$R_{180}$&$R_{270}$&V        &H        &D        &D'  \\
$R_{90}$ &$R_{90}$ &$R_{180}$&$R_{270}$&$R_0$    &D'       &D        &V        &H   \\
$R_{180}$&$R_{180}$&$R_{270}$&$R_0$    &$R_{90}$ &H        &V        &D'       &D   \\
$R_{270}$&$R_{270}$&$R_0$    &$R_{90}$ &$R_{180}$&D        &D'       &H        &V   \\
V        &V        &D'       &H        &D        &$R_0$    &$R_{180}$&$R_{90}$ &$R_{270}$\\
H        &H        &D        &V        &D'       &$R_{180}$&$R_0$    &$R_{270}$&$R_{90}$ \\
D        &D        &H        &D'       &V        &$R_{270}$&$R_{90}$ &$R_0$    &$R_{180}$\\
D'       &D'       &V        &D        &H        &$R_{90}$ &$R_{180}$&$R_{90}$ &$R_0$

\end{tabular}
\end{table}

\begin{table}[h]
  \centering
\caption{$Q_8$}
\label{Quaternion}
\begin{tabular}{r|rrrrrrrr}
*  &  1 & -1 &  i & -i &  j & -j &  k & -k    \\
\hline
1  &  1 & -1 &  i & -i &  j & -j &  k & -k \\
-1 & -1 &  1 & -i &  i & -j &  j & -k &  k \\
i  &  i & -i & -1 &  1 &  k & -k & -j &  j \\
-i & -i &  i &  1 & -1 & -k &  k &  j & -j \\
j  &  j & -j & -k &  k & -1 &  1 &  i & -i \\
-j & -j &  j &  k & -k &  1 & -1 & -i &  i \\
k  &  k & -k &  j & -j & -i &  i & -1 &  1 \\
-k & -k &  k & -j &  j &  i & -i &  1 & -1 \\
\end{tabular}
\end{table}
\end{itemize}

\end{document}
