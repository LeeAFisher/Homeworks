\documentclass[12pt]{report}
\usepackage[margin=0.5in]{geometry}
\usepackage{amssymb,textcomp,amsmath,tikz}


\begin{document}

\textbf{Algebra 1 Homework 5}\\
\indent \textbf{Lee Fisher}\\
\indent \textbf{2017-09-24}

\vspace{0.3cm}

\noindent {\bf 3.1 \#3:} Proof. Let $a$ and $b$ be elements of $G$. Then
\begin{align*}
   \textrm{Inn}(ab) &= \phi_(ab)\\
                    &= ab [.] (ab)^{-1}\\
                    &= ab [.] b^{-1}a^{-1}\\
                    &= \textrm{Inn}(a) \circ \textrm{Inn}(b)
\end{align*}

Since $Z(G)$ is the set $\{ a \in A | gag^{-1} = a\}$, we have that $a \in Z(G)$
if and only if $\textrm{Inn}(a) = Id_G$. This is because $gag^{-1} = a$ for all
$g$ if and only if $g = aga^{-1}$ for all $g$.\\

Part 2. Any automorphism of $D_8$ must respect the group operation. Thus any
automorphism is defined by its actions on the generators. Let $\phi \in
\textrm{Aut}(D_8)$, then $\phi(r)$ and $\phi(s)$ completely define $\phi$.
We know $\phi$ sends elements to elements of the same order. Thus $\phi(r)$ can
be $r$ or $r^3$, and $\phi(s)$ can be one of $s$, $r^2$, $sr$, $sr^2$ or $sr^3$.
Now we know $|\textrm{Aut}(D_8)| \leq 10$.

To compute $\textrm{Aut}(D_8)$ we can use the two isomorphisms: $\rho$ and
$\phi$. The are defined with $\rho(r) = r$ and $\rho(s) = sr$; and $\phi(r) =
r^3$ and $\phi(s) = s$. We can see that $|\rho| = 4$ and $|\phi|=2$ this tells
us that $|\textrm{Aut}(D_8)|$ is either 8 or 4, and because no composition of
$\rho$ creates the permutation $\phi$ we know that $|\textrm{Aut}(D_8)| = 8$.
A quick computation shows that $\rho\circ\phi(r)=\phi\circ\rho^{-1}(r)=r^3$
and that $\rho\circ\phi(s)=\phi\circ\rho^{-1}(s)=sr$. So $\textrm{Aut}(D_8)
\cong D_8$ and $\textrm{Aut}(D_8)$ is generated by $\phi$ and $\rho$.

Now let's compute $\textrm{Inn}(D_8)$. We know $Z(D_8) = \langle r^2 \rangle$ so
if we consider the homomorphism $\phi: D_8 \to \textrm{Inn}(D_8)$. We have by
the first isomorphism theorem that $\textrm{Inn}(D_8) \cong D_8 / \langle r^2
\rangle$. This group is isomorphic to the Klein 4-Group.


\noindent {\bf 3.4.} Let $\phi : G \to \bar{G}$ be an epimorphism, let $N =
ker(\phi)$ and let $H$ be a subgroup of $G$ containing $N$.

$\rightarrow$ Suppose $H$ is normal in $G$. We know for all $g$, $gHg^{-1} = H$.
Also, since $\phi$ is an epimorphism any element $\bar{g} in \bar{G}$ satisfies
$\bar{g} = \phi(g)$ for at least one $g \in G$. Consider $\bar{g}\phi(H)\bar{g}
^{-1}$ there is a $g$ for which this equals $\phi(g)\phi(H)\phi(g)^{-1} =
\phi(gHg^{-1}) = \phi(H)$. So we know that if $H$ is normal then $\phi(H)$ must
be normal.

$\leftarrow$ Suppose $\phi(H)$ is normal in $\bar{G}$. Since $\phi$ is an
epimorphism we can write $\phi(H) = \bar{g}\phi(H)\bar{g}^{-1} = \phi(gHg^{-1})$
This implies that $gHg^{-1} = nH$, that the preimages are equal up to an element
from $N$. However, $N \leq H$ so $nH = H$, and $gHg^{-1} = H$.Thus $H$ is normal
and we're done.

For the next part: we need to show that that the lattice subgroups of $G$ that
contain $N$ is in bijection with the subgroup lattice of $\bar{G}$. We know from
the first isomorphism theorem that $G/N \cong \bar{G}$.

Let $\pi$ be the natural projection from $G$ to $G/N$. If $H$ is a subgroup such
that $N \leq H \leq G$ then $\pi(H)$ will be a subgroup of $G/N$ because $\pi$
is a homomorphism. Likewise $\phi(H)$ will be a subgroup of $\bar{G}$ because
the groups are isomorphic. Now suppose $\bar{S} \leq \bar{G}$ then $\bar{S}$ is
isomorphic to $S$ a subgroup of $G/N$. Consider $\pi^{-1}(S)$, this is the group
$SN$ which is a subgroup of $G$ that contains $N$.

Last part: Let $B \leq A$ be subgroups of $G$ containing $N$. Let $aB$ be an
element of the set $A/B$. We know that each element in $A/B$ maps to a coset of
$\phi(A)/\phi(B)$, because $\phi$ is surjective. However the preimage, $\phi^
{-1}(\bar{a}\phi(B))$, is a coset in $A/B$ since $A$ and $B$ contain $N$. So
$|A:B| = |\phi(A):\phi(B)|$ and we're done.

\noindent {\bf 3.5.}

First part: Suppose $H$ is a normal subgroup of $G$ with prime index $p$ and $K$
is a subgroup of $G$ but is not a subgroup of $H$. Let $\pi$ be the natural map
from $G$ to $G/H$. We know that $H \leq HK \leq G$ This tells us that
$$\pi(HK) \cong HK/H \leq G/H.$$

Then by the second isomorphism theorem we get that $K/(H \cap K) \leq G/H$.
Since $K$ is not a subgroup of $H$ and $G/H$ has prime order. This extends to:
$$KH/H \cong K/(H\cap K) \cong G/H.$$

From this relation we have $|K : H \cap K| = p$ immediately. To see that $G
\cong HK$ is not so hard. This follows from the fourth isomorphism theorem,
because $\pi$ is an epimorphism and $G$ and $HK$ are both subgroups of $G$ that
contain $H$. Since their projections are isomorphic, they themselves must be
isomorphic.

Second part: Suppose $H$ is a normal subgroup of $G$ with prime index $p$ and
$K$ is a subgroup of $G$ with $|K: K\cap H| \neq p$. The same relations as
before hold from the second isomorphism theorem.

$$KH/H \cong K/(H \cap K) \leq G/H $$

But this time since $|K : K\cap H| \neq p$ we know that $K/(H \cap K)$ must be
trivial. So this means $H \cap K = K$, or $K \leq H$.

\noindent {\bf 3.6.}

1. Suppose $G$ and $G'$ are groups, let multiplication in $G \times G'$ be
defined as $(g,g')(h,h') = (gh,g'h')$. This makes $G \times G'$ a group.

\begin{itemize}
  \item Closure: Let $(g,g')(h,h') = (gh,g'h')$ for any $g,h \in G$ and $g',h'
        \in G'$. Since $G$ and $G'$ are both closed, $(gh,g'h') \in G\times G'$.

  \item Associativity: $(g,g')((h,h')(k,k')) = (g,g')(hk,h'k') = (ghk,g'h'k')$
        by $G$ and $G'$ being associative. Also $((g,g')(h,h'))(k,k') =
        (gh,g'h')(k,k') = (ghk,g'h'k')$ again because $G$ and $G'$ are
        associative. Thus $G \times G'$ is associative.

  \item If $e$ is the identity in $G$ and $e'$ is the identity in $G'$ then
        for all $(g,g')$ in $G \times G'$ we have $(e,e')(g,g') = (eg,e'g') =
        (g,g')$ and also $(g,g')(e,e') = (ge,g'e') =
        (g,g')$. So $G \times G'$ has an identity element.

  \item We have $(g,g')^{-1} = (g^{-1},g'^{-1})$. So inverses exist.
  \end{itemize}

Now we know that $G \times G'$ is a group.

2. Let $M$ and $N$ be normal subgroups of $G$ such that $G = MN$. We define a
homomorphism $\phi: G \to G/M \times G/N$ by $\phi(g) = (gM, gN)$. The kernel of
this homomorphism is $M \cap N$. if $x \in M \cap N$ then $\phi(x) = (M,N)$
which is the identity element of $G/M \times G/N$.

\noindent {\bf Pg 87 \#17.}

\begin{itemize}
  \item (a) The order of $D_16$ is $16$ and the order of $\langle r^4 \rangle$
        is $2$. So $|\bar{G}| = |D_16/\langle r^4 \rangle| = 16/2 = 8$.

  \item (b) Any element in $\bar{G}$ is of the form $\bar{s}^a\bar{r}^b$ where
        $a$ is either $1$ or $0$ and $b$ is one of $0,1,2,3$. The order of the
        element $s$ does not change in $\bar{G}$. The order of $r$ is now $4$
        instead of $8$ since $r^4$ has been modded out.

  \item (c) (I will omit the bars, these are all elements in $\bar{G}$ though)
        ${|s| = 2, |sr| = 2, |sr^2| = 2, |sr^4| = 2, |r| = 4, |r^2| = 2, |r^3|
         = 4, |1| = 1}$

  \item (d) $\bar{rs} = \bar{s}\bar{r}^3$, $\bar{sr^{-2}s} = \bar{r}^2$, and
        $\bar{sr^{-1}sr} = \bar{r}^2$

  \item (e) Omitting the bars: $rH = \{ rs, r^3, rsr^2, r \}= \{ sr^3, r^3, sr,
        r\} = Hr$. And for conjugation by $s$, $sH = \{ 1, sr^2, r^2, s \}=
        \{ 1, r^2s, r^2, s\} = Hs$. So $H$ is fixed under conjugation by the
        generators, so $H$ is normal. Each non-identity element of $H$ has order
        $2$, and $H$ has order $4$, thus $H$ is the Klein-4 group.

        The preimage of $H$ in $G$ is $\{ 1, r^2, r^4, r^6,s,sr^2,sr^4,sr^6 \}$.
        This is isomorphic to $D_8$. with isomorphism $\phi(s) = s$ and
        $\phi(r^2) = r$.

  \item The center of $\bar{G}$ is $\langle \bar{r}^2 \rangle$. Also $\bar{G}/Z
         (\bar{G})$ is isomorphic to the Klein 4-Group. This is because
         $\bar{G}$ is isomorphic to $D_4$ with isomorphism $\phi(\bar{r}) = r$
         and $\phi(\bar{s}) = s$. In one of the previous problems we showed that
         $D_8/Z(D_8) \cong K_4$.


\end{itemize}

\noindent {\bf Pg 88 \#32.} The subgroups $Q_8$ and $\langle 1 \rangle$ are
trivially normal. The subgroups $\langle i \rangle$, $\langle j \rangle$
$\langle k \rangle$ all have index 2 and thus are normal. $Q_8$ mod each one is
isomorphic to $Z_2$ the only group of order $2$. The subgroup $\{ 1, -1 \}$ is
normal as well because it is the center of $Q_8$. $Q_8 / \langle -1 \rangle
\cong Z_2 \times Z_2$. This is $i^2, j^2, k^2 = -1$, so $i$, $j$ and $k$ all
have order $2$. 



\end{document}
