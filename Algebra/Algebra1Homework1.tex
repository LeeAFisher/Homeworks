\documentclass[12pt]{report}
\usepackage[margin=0.5in]{geometry}
\usepackage{amssymb,textcomp}
\title{\textbf{Algebra I: Homework 1}}
\author{Lee Fisher}
\date{}
\begin{document}

\textbf{Algebra 1 Homework 1}\\
\indent \textbf{Lee Fisher}\\
\indent \textbf{2017-08-22}

\begin{itemize}
\item Proposition 0.1: Let $f:A \to B$ be a function. Then:
\begin{enumerate}
\item $f$ is injective iff $f$ has a left inverse.
\item $f$ is surjective iff $f$ has a right inverse.
\item $f$ is bijective iff $f$ admits a left and a right inverse. In this case the inverses are equal and the unique.
      This function is called the inverse of $f$ and is denoted by $f^{-1}$
\item If $A$ and $B$ are finite sets of the same order, $f$ is bijective if and only if $f$ is injective if and only if
      $f$ is surjective.
\end{enumerate}

\textit{Proof of 1.}\\
$\rightarrow$ Suppose $f$ is injective. By definition this means that for any $b \in f(A)$ there exists a unique $a$
              such that $f^{-1}(b) = a$. We define $g(b) = f^{-1}(b)$ if $b$ is in $f(A)$ and if $b$ is not in $f(A)$
              then $g(b)$ is arbitrary. In this way $g \circ f = id_A$, because $g \circ f(a) = a$ for all $a \in A$.

$\leftarrow$ Suppose $f$ has a left inverse. This means, by definition, there is a function $g:B \to A$ such that
             $g\circ f = id_A$. Consider two elements in $A$, $a_1$ and $a_2$ with $a_1 \neq a_2$. (If $|A|=1$ then $f$
             is trivially injective.) $f(a_1) \neq f(a_2)$, otherwise this would mean $g(f(a_1)) = a_1$ and
             $g(f(a_1)) = a_2$ which contradicts the definition of a function. Therefore $f(a_1) \neq f(a_2)$ and $f$ is
             injective. $\square$\\

\textit{Proof of 2.}\\
$\rightarrow$ Suppose $f$ is surjective this means $f(A) = B$, so for any $b \in B$ there is at least one $a \in A$ such
              that $f(a) = b$. Consider $g:B \to A$ where for all $b$ we choose $g(b)$ so that $g(b) \in f^{-1}(b)$. In
              this way $f \circ g = id_B$, because for any $b \in B$, $f \circ g(b) = b$. Thus $f$ has a right inverse.

$\leftarrow$ Suppose for sake of contradiction that $f$ has a right inverse and $f$ is not surjective. This means, by
             definition, there is a function $g$ such that $f \circ g = id_B$ and that there is also an element $b$ in
             $B$ such that there is no $a$ in $A$ where $f(a) = b$. Here's the contradiction, in this case
             $f(g(b)) \neq b$ which contradicts $f \circ g = id_B$. Therefore $f$ is surjective. $\square$\\

\textit{Proof of 3.}\\
Suppose $f$ is bijective. By definition $f$ is both injective and surjective. Since $f$ is injective (from the proof of
1) $f$ has a left inverse, and since $f$ is surjective (from the proof of 2) $f$ has a right inverse. Likewise suppose
$f$ has both a left and right inverse. Then from the two previous proofs, $f$ is both injective and surjective, and
therefore bijective.

We call the left inverse as $f^{-1}_L$ and the right inverse as $f^{-1}_R$. Consider for an element $b \in B$,
$f^{-1}_L \circ f \circ f^{-1}_R (b)$. Well $f\circ f^{-1}_R=id_B$, so $f^{-1}_L\circ f\circ f^{-1}_R (b)=f_L^{-1}(b).$
Also $f^{-1}_L \circ f = id_A$ so $f^{-1}_L \circ f \circ f^{-1}_R (b)=f_R^{-1}(b).$ Therefore $f_R^{-1}(b)=f_L^{-1}(b)$
and the left and right inverses are equal.

Now we know that if $f$ is bijective then any left inverse will also be a right inverse. Consider two inverses of $f$,
$f^{-1}_1$ and $f^{-1}_2$. We know that for any $b \in B$, $f\circ f_1^{-1}(b) = f\circ f_2^{-1}(b)$. Therefore
$f_1^{-1}(b) = f_2^{-1}(b)$, so the inverses are the same, and thus unique.$\square$\\

\textit{Proof of 4.}\\
Suppose $A$ and $B$ are finite sets of the same size and $f$ is a function from $A$ to $B$. We want to prove that
bijectivity, surjectivity, and injectivity are equivalent. From the definition of bijectivity, it will suffice to prove
that a function is surjective if and only if it is injective.

\begin{enumerate}
\item Surjectivity $\to$ Injectivity. Suppose $f:A \to B$ is surjective. For sake of contradiction suppose $f$ is not
      injective. This means there are two distinct numbers $a_1$ and $a_2$ such that $f(a_1) = f(a_2)$. Since the
      since the cardinality of $A$ and $B$ are the same, this means the image of $f(A)$ is a strict subset of $B$. Which
      contradicts $f$ being surjective. Therefore $f$ must be injective.

\item Injectivity $\to$ Surjectivity. Suppose $f:A \to B$ is injective. For sake of contradiction suppose $f$ is not
      surjective. This means $f(A)$ is strictly contained in $B$, however we know $A$ and $B$ are the same size. This
      means there must be two elements of $A$ that map to the same thing in $B$. Thus contradicting injectivity. Thus
      $f$ must be surjective. $\square$\\
\end{enumerate}

\item Proposition 0.2 Let $\sim$ be an equivalence relation of the set $A$. For any $a,b \in A,$
\begin{enumerate}
  \item $a \sim b$ if and only if $\bar{a} = \bar{b}$
  \item if $\bar{a} \neq \bar{b}$ then $\bar{a} \cap \bar{b} = \emptyset$
\end{enumerate}

\textit{Proof 1.}\\
$\rightarrow$ Suppose $a\sim b$ and consider $\bar{a}$ and $\bar{b}$. Since $a \sim b$ we know that $\bar{a} \subset \bar{b}$
              because if any element is equivalent to $a$ it must also be equivalent to $b$, and likewise that since
              $b \sim a$ we know that $\bar{b} \subset \bar{a}$. Therefore $\bar{a} = \bar{b}$.\\
$\leftarrow$ Suppose $\bar{a} = \bar{b}$. This means $a \in \bar{b}$ and $b \in \bar{a}$. Therefore $a \sim b$.

\textit{Proof 2.}\\
Suppose $\bar{a} \neq \bar{b}$ and for sake of contradiction, suppose there is an element $c$ that is in $\bar{a} \cap
\bar{b}$. This means that $a \sim c$ and $b \sim c$ and that by the previous logic, $\bar{a} = \bar{c}$, $\bar{b}= \bar{c}$,
and $\bar{a} = \bar{b}$. Which is a contradiction of $\bar{a} \neq \bar{b}$. Therefore the intersection must be empty.
$\square$\\

\item Proposition 0.6 Let $n$ be a fixed positive integer. Then $$(\mathbb{Z}/n\mathbb{Z})^X = \{\bar{a} \in
      \mathbb{Z}/n\mathbb{Z}|1 \leq a < n \textnormal{ and $a,n$ are relatively prime}\}.$$

\textit{Proof.}\\
Consider an element $a$ in $(\mathbb{Z}/n\mathbb{Z})^X$. This means there is some number $a^{-1}$ for which
$a^{-1}a = 1$ mod $n$. In other words there is some multiple of $n$, $mn$ for which $a^{-1}a + mn = 1$ where all numbers
are integers. We arrive at the conclusion that $a$ and $n$ are relatively prime. If they had a common divisor then an
integer linear combination of $a$ and $n$ would also be divisible by that number; the equation $xa + yn = 1$ would have
no solutions. In the other direction suppose $a$ and $n$ are relatively prime. This means $gcd(a,n) = 1$ and since the
$gcd$ is a linear combination. There exist solutions $x$ and $y$ such that $xa + yn = 1$. This equation mod$n$ tells us
that $a^{-1} = x$ and that $a$ is in $(\mathbb{Z}/n\mathbb{Z})^X$.


\end{itemize}

\end{document}
