\documentclass[12pt]{report}
\usepackage[margin=0.5in]{geometry}
\usepackage{amssymb,textcomp,amsmath,tikz}


\begin{document}

\textbf{Algebra 1 Homework 7}\\
\indent \textbf{Lee Fisher}\\
\indent \textbf{2017-10-8}

\vspace{0.3cm}

\noindent{\bf 5.2 \#1:} Find the isomorphism classes of abelian groups of order
200.

Here are the isomorphism classes, from the fundamental theorem of finitely
generated Abelian Groups:

\begin{align*}
  G &\cong \mathbb{Z}_{200}\\
    &\textrm{ or } \mathbb{Z}_{100} \times \mathbb{Z}_2\\
    &\textrm{ or } \mathbb{Z}_{50} \times \mathbb{Z}_2 \times \mathbb{Z}_2\\
    &\textrm{ or } \mathbb{Z}_{40} \times \mathbb{Z}_5\\
    &\textrm{ or } \mathbb{Z}_{20} \times \mathbb{Z}_{10}\\
    &\textrm{ or } \mathbb{Z}_{10} \times \mathbb{Z}_{10} \times \mathbb{Z}_2\\
\end{align*}

From the divisibility restriction, these are all of the factors.\\

\noindent{\bf 5.2 \#2:}
Find the invariant factors and the elementary divisiors of the abelian group.

$$G = \mathbb{Z}_2 \times \mathbb{Z}_2 \times\mathbb{Z}_2 \times\mathbb{Z}_9
\times \mathbb{Z}_5 \times\mathbb{Z}_5.$$

For finding the elementary divisors we can see that $G$ has been decomposed into
a direct product of prime power cyclic subgroups. So the elementary divisors are
in this order: $2$, $2$, $2$, $9$, $5$, and $5$.

To find the invariant factors we can combine cyclic groups that share no common
factors to be the cyclic group on the product, then we have that

$$G \cong \mathbb{Z}_{90} \times \mathbb{Z}_{10} \times \mathbb{Z}_2.$$

Which concludes this problem.\\

\noindent{\bf 5.2 \#4:} Let $G$ be a finite group and $p$ a prime factor of
$|G|$. Prove that the number of order $p$ elements in $G$ is congruent to $-1$
modulo $p$.\\

Consider solutions in $G$ to $x_1x_2\cdots x_p = 1$, There are $|G|^{p-1}$ such
solutions. That is you can choose any element from $G$ that you want from the
first $p-1$ elements and you must choose the unique inverse of the product of
those elements to be $x_p$. Since $p$ divides $|G|$, $p$ also divides
$|G|^{p-1}$. If $(x_1,x_2,\dots,x_p)$ is a solution then any cyclic permutation
of $(x_1,x_2,\dots x_p)$ is a solution because if you have one solution then you
have $x_2\cdots x_p = x_1^{-1}$ which means that $x_2\cdots x_px_1 = 1$ is also
a solution.

If $(x_1,x_2,\dots,x_p)$ is a solution where at least two elements differ then
the cyclic permutations of this solution give you $p$ different solutions. So
the number of solutions to $x_1x_2\cdots x_p = 1$ with not all $x_i$ equal is a
multiple of $p$. Therefore the number of solutions with $x_i$ all equal is a
difference of two multiples of $p$, this means it must be a multiple of $p$.

The number of solutions to $x^p = 1$ is congruent to $0$ mod $p$.  Solutions to
this equation are order $p$ elements, since $p$ is a prime, and the identity
element. So the number of order $p$ elements is congruent to $-1$ mod $p$.\\

\noindent{\bf 5.3 \#2:}
Let $G$ be a finite group and $N_1,..., N_n$ normal subgroups of $G$ such that
$G=N_1 \cdots N_n$ and $|G| = |N_1| \cdots |N_n|$. Prove that $G$ is the
internal direct product of $G$.\\

We know that $G$ is the internal direct product of all the $N_i$ if and only if
it is isomorphic to the external direct product of the $N_i$. Consider the
map:

$$\phi: N_1 \times N_2 \dots \times N_n \to G$$
By $\phi((m_1,\dots,m_n)) = m_1m_2\dots m_n$. $\phi$ is a homomorphism because,
\begin{align*}
\phi((m_1,\dots m_n)(k_1,\dots,k_n)) &=\phi(m_1k_1,m_2k_2,\dots m_nk_k)\\
                                     &=m_1k_1\cdots m_nk_n\\
\textrm{Since the $N_i$ are normal:} &=m_1\cdots m_nk_1\cdots k_n\\
                                     &=\phi(m_1,\dots,m_n)\phi(k_1,\dots,k_n)
\end{align*}

This homomorphism is surjective because $G = N_1 \cdots N_n$. Suppose for sake
of contradiction that $\phi$ is not 1-1, this would mean that two different
elements in the external direct product multiply to be the same element in $G$,
and since both groups are finite and $\phi$ is surjective, this implies that
$|N_1||N_2| \cdots |N_n| = |N_1 \times N_2 \cdots N_n| > |G|$, but this
contradicts the assumption that $|G| = |N_1||N_2|\cdots |N_n|$. So $\phi$ must
be 1-1, and therefore, an isomorphism. So $G$ is isomorphic to the external
direct product, and must be the internal direct product of the $N_i$.\\

\noindent{\bf 5.5 \#1:} Prove Proposition $5.16$.\\

{\bf Proposition 5.16}
Let $G$ be a group, $H$, $K$ subgroups of $G$, and $H \trianglelefteq G$. Let
$\varphi: K \to Aut(H)$ be the homomorphism associated with the conjugate action
 of $K$ on $H$. Then the following statements are equivalent:
\begin{enumerate}
      \item $\phi: H \rtimes_{\varphi} K \to G$ defined by $\phi(h,k) = hk$ is
      an isomorphism.
      \item Every element $g \in G$ can be written as $g = hk$ with $h \in H$
      and $k \in K$ in a unique way.
      \item $G = HK$ and $H \cap K = \{ e\}$.
\end{enumerate}

$1 \to 2$ Since $\phi$ is an isomorphism from $H \rtimes_{\varphi} K \to G$
every element in $G$ is a unique product of elements in the semi-direct product.
But every element in the semi-direct product is an element in $H \times K$,
(only the multiplication operation is different, not the elements) so every
element $g \in G$ is a unique product of elements $hk$ with $h \in H$ and $k \in
K$.

$2 \to 3$ Since every element in $G$ can be written as $g = hk$ with $h \in H$
and $k \in K$ this gives us that $G = HK$. Also suppose there is an $a H \cap K$
with $a \neq e$. This means that if $g = hak$ then $g = (ha)k = h(ak)$, which
contradicts the uniqueness of the representation of $g$ as a product of elements
in $H$ and $K$. So $H \cap K = \{ e\}$.

$3 \to 1$ Suppose $G = HK$ and $H \cap K = \{ e\}$. Let consider a map $\phi$
which sends $H \rtimes_{\varphi} K$ to $G$ by $\phi((h,k)) = hk$.

This is a homomorphism because:
\begin{align*}
\phi((h_1,k_1)(h_2,k_2)) &= \phi((h_1k_1h_2k_1^{-1},k_1k_2))\\
                         &= h_1k_1h_2k_1^{-1}k_1k_2\\
                         &= h_1k_1h_2k_2\\
                         &= \phi((h_1,k_1))\phi((h_2,k_2))
\end{align*}

This homomorphism is surjective because $G = HK$. Now consider $Ker(\phi)$,
these are elements $h \in H$ and $k \in K$ such that $hk = 1$ these are elements
in $H$ and $K$ separately whose inverses lie in the other group. This means one
of the pairs must lie in the subgroup $H \cap K$ and the other element must be
it's inverse. However, $H \cap K$ is just the identity element, so $Ker(\phi)$
is just the element $(e,e)$. Thus $\phi$ is also 1-1, and therefore it must be
an isomorphism.\\

\noindent{\bf 5.5 \#4 (a):} For any positive integer $n$, prove that
$Aut(\mathbb{Z}_n) \cong \mathbb{Z}_n^{\times}$.\\

Let $\phi$ be an automorphism of $Z_{n}$. Since $\phi$ is a homomorphism, if $m
\in \mathbb{Z}_n$ then $\phi(m*1) = m\phi(1)$. So all of the homomorphisms are
completely determined by where they send the element $1$ in $\mathbb{Z}_n$.
In order for it to be an automorphism it is necessary that is sends the element
$1$, which generates $\mathbb{Z}_n$, to another generator of $\mathbb{Z}_n$.
Also since cyclic groups are completely determined by their only generator, this
is a sufficient condition as well. Thus: $\phi \in Aut(\mathbb{Z}_n) \iff
\phi(1) \in \mathbb{Z}_n^{\times}$.

The map $\varphi$ that sends $\phi(x) = gx \in Aut(\mathbb{Z}_n)$ to $g \in
\mathbb{Z}_n^{\times}$ is one to one and onto because of the previous argument.
It is homomorphism as well: $\varphi(\phi_1 \circ \phi_2) = \varphi(g_1g_2x) =
g_1g_2 = \varphi(\phi_1)\varphi(\phi_2)$.\\

\noindent{\bf 5.5 \#4 (b):} For any primes $p < q$, if $p \text{ }|\text{ }q-1$,
there exists a monomorphism $\varphi: \mathbb{Z}_p \to Aut(\mathbb{Z}_q)$ and
$\mathbb{Z}_q  \rtimes_{\varphi}\mathbb{Z}_p$ is a non-abelian group of order
$pq$.\\

Since $Aut(\mathbb{Z}_q)$ is isomorphic to $\mathbb{Z}_q^{\times}$ and $p$ is a
prime divisor of $q-1 = |\mathbb{Z}_q^{\times}|$, there is an element $g$ of
order $p$ in $Aut(\mathbb{Z}_q)$. Consider the map:

$\varphi: \mathbb{Z}_p \to Aut(\mathbb{Z}_q)$

By $\varphi(n) = g^n$. This is a homomorphism because $\varphi(n_1+n_2) =
g^{n_1+n_2} = g^{n_1}g^{n_2} = \varphi(n_1)\varphi(n_2)$. Also consider
$Ker(\phi)$, these are $n \in \mathbb{Z}_p$ such that $g^n = Id$. Since $g$ has
order $p$, this means $Ker(\phi) = \{ 0\}$, so the map is one to one.

We also know that since $\varphi$ is a nontrivial homomorphism, we know that
$\mathbb{Z}_q  \rtimes_{\varphi}\mathbb{Z}_p$ is a non-abelian group of order
$pq$. Because the semidirect product of two Abelian groups is only Abelian if
the homomorphism is trivial, and the order the semidirect product is the product
of the orders of the groups.\\

\noindent{\bf Page 186 \#11}:
Classify groups of order $28$ (there are four isomorphism types).

So, $28 = 2^2 \cdot 7$. Consider the number of Sylow 7-subgroups, $n_7$. We have
$n_7 = 1$ mod $7$ and $n_7 | 4$ so $n_7 = 1$. This tells us that a group of
order $28$ is a semidirect product of a group of order $4$ with the cyclic group
of order $7$.

If $H$ is a group of order $4$ and $\varphi: H \to Aut(\mathbb{Z}_7)$. Since the
image of $|H|/|Ker(\varphi)|$ must a divisor of $Aut(\mathbb{Z}_7)$ which has
order $6$, we know that $|Ker(\varphi)| = 2$ or the Kernel is trivial. If the
kernel is trivial then we have that $G = \mathbb{Z}_2 \times \mathbb{Z}_2 \times
\mathbb{Z}_7$ or that $G =\mathbb{Z}_4 \times \mathbb{Z}_7$. If the kernel of
$\varphi$ has order $2$ then we have two more cases. Either $H = \mathbb{Z}_2
\times \mathbb{Z}_2$ or $H = \mathbb{Z}_4$.

For the first case we note that inversion (multiplication by 6) is the only
automorphism of $\mathbb{Z}_7$ with order $2$. Then we consider a map from $H$
to $Aut(\mathbb{Z}_7)$ with a Kernel of order $2$. Then we have $\varphi: (0,0)
\to Id$ and $\varphi$ sends two of $(1,0)$ $(0,1)$ or $(1,1)$ to $-Id$, and the
last one of the $3$ goes to $Id$ again. However permuting $(1,0)$, $(1,1)$, and
$(0,1)$ are all automorphisms of $\mathbb{Z}_2 \times \mathbb{Z}_2$ so it
does not matter which two of the elements go to $-Id$, because the semidirect
products defined by each choice will all be isomorphic.

Now we consider the homomorphism from $\mathbb{Z}_4 \to \mathbb{Z}_7$. This map
must be $\varphi(0) = \varphi(2) = Id$ and $\varphi(1) = \varphi(3) = -Id$. With
all of this said there are four non-isomorphic groups of order $28$ and they are:

$\mathbb{Z}_4 \rtimes_{\varphi} \mathbb{Z}_7$,
$(\mathbb{Z}_2 \times \mathbb{Z}_2) \rtimes_{\varphi} \mathbb{Z}_7$,
$\mathbb{Z}_4 \times \mathbb{Z}_7$, and
$\mathbb{Z}_2 \times \mathbb{Z}_2 \times \mathbb{Z}_7$.

\end{document}
