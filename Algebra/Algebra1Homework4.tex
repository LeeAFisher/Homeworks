\documentclass[12pt]{report}
\usepackage[margin=0.5in]{geometry}
\usepackage{amssymb,textcomp,amsmath,tikz}


\begin{document}

\textbf{Algebra 1 Homework 4}\\
\indent \textbf{Lee Fisher}\\
\indent \textbf{2017-09-17}

\vspace{0.3cm}

\noindent {\bf 2.7 \#1:}

First we need to show that $N$ is a normal subgroup of $G$.
For a fixed $g \in G$, consider the coset $gN$. Elements in $gN$ are of the form
$gn$. Where $n$ has the property that for any $x \in G$ there is an $h \in H$
with $n = xhx^{-1}$. Let $x = g^{-1}$, then we have $n = g^{-1}h_*g$, and also
that $gn = h_*g$ for some $h_* \in H$.\\

To conclude we need to prove that $h_* \in \cap xHx^{-1} =N$ then we'll have $gn
= h_*g = eh_*e^{-1}g \in Ng$. We know $h_* = gng^{-1}$ where $n \in N$. Consider
a $y \in G$ then $yh_*y^{-1} = ygng^{-1}y^{-1} = (yg)n(yg)^{-1}$. However, we
know $yg$ is an arbitrary element in $G$ and $n \in eHe^{-1} = H$. So we
conclude, $h_* \in N$ and $gn = h_*g \in Ng$. So $gN = Ng$ and $N$ is normal in
$G$.\\

Now to prove that $N$ is the largest normal subgroup contained in $H$. Let $M$
be a normal subgroup of $G$ contained in $H$. Consider $m \in M$ we know that
$m \in H$ by supposition and that because $M$ is normal, for all $g \in G$,
$gmg^{-1} \in M$. So $m \in N$. Any normal subgroup contained in $H$ must be a
subset of $N$.\\


\noindent {\bf 2.7 \#2.}

\begin{enumerate}
  \item \begin{itemize}
        \item Reflexivity: $a \sim a$ because $a \in HaK$. Because $H$ and $K$
              are both subgroups and $eae \in HaK$.
        \item Symmetry: Suppose $a \sim b$. This means $a \in HbK$. For some $h
              \in H$ and $k \in K$, $a = hbk$. With some algebra: $h^{-1}ak^{-1}
               = b$, so $b \in HaK$.
        \item Transitivity: Suppose $a \sim b$ and $b \sim c$. So $a \in HbK$
              and $b \in HcK$, so $a = h_1bk_1$ and $b = h_2ck_2$, therefore
              $a = (h_1h_2)c(k_1k_2)$, and $a \in HcK$. So $a \sim c$ and we're
              done.
        \end{itemize}
  \item Suppose $f(aW) = f(bW)$
        \begin{align*}
          f(aW) &= f(bW) \\
           aWxK &= bWxK \\
        \textrm{This implies $\exists e,k_1 \in K$ such that:} &\\
            aWx &= bWxk_1 \\
        \textrm{As well $\exists e,w \in W$ such that:} &\\
            awx &= bxk_1 \\
            a(xk_2x^{-1})x &= bxk_1 \\
            axk_2 &= bxk_1 \\
            b^{-1}axk_2 &=xk_1 \\
            b^{-1}ax &= xk_1k_2^{-1}\\
            b^{-1}a &= xk_1k_2^{-1}x^{-1}\\
            &\to b^{-1}a \in xKx^{-1}
        \end{align*}

        We also know by supposition that $a$ and $b$ are in $H$, and that since
        $H$ is a subgroup $b^{-1}a$ is also in $H$. So $b^{-1}a \in W$ $\to$ $aW
        = bW$. So we know $f$ is injective.\\

        Next part: Let $L$ be the set of representatives of left cosets of $W$
        in $H$. Consider: $L_h = \{ hxK |  h \in L\}$. We need $L_{h_1} \cap
        L_{h_2} = \emptyset$ when $h_1 \neq h_2$ and $\cup L_h = HxK$.\\

        First, suppose $a \in L_{h_1}$ and $a \in L_{h_2}$. This means that $a =
        h_1xk_1$ and $a = h_2xk_2$. So $h_1xk_1 = h_2xk_2$ implies $h_2^{-1}h_1=
        xk_2k_1^{-1}x^{-1}$. This means $h_1$ and $h_2$ lie in the same left
        coset of $W$. So we know the $L_h$ are disjoint.\\

        Secondly, suppose $a \in HxK$. This means $a = hxk$, thus $a \in L_h$.
        So for every $a \in HxK$ there is an $L_h$ that contains $a$. So $\cup
        L_h = HxK$. Also $\cup L_h$ cannot exceed $HxK$ because each $L_h$
        contains only elements from $HxK$.\\

        Last part (this was mostly copy + pasted): Let $R$ be the set of
        representatives of right cosets of $M = x^{-1}Hx \cap K$ in $K$.
        Consider: $R_k = \{ Hxk |  k \in R\}$. We need $R_{k_1} \cap R_{k_2} =
        \emptyset$ when $k_1 \neq k_2$ and $\cup R_k = HxK$.\\

        First, suppose $a \in R_{k_1}$ and $a \in L_{k_2}$. This means that $a =
        h_1xk_1$ and $a = h_2xk_2$. So $h_1xk_1 = h_2xk_2$ implies $k_1k_2^{-1}
        = x^{-1}h_1^{-1}x$. This means $k_1$ and $k_2$ lie in the same right
        coset of $M$. So we know the $R_k$ are disjoint.\\

        Secondly, suppose $a \in HxK$. This means $a = hxk$, thus $a \in R_k$.
        So for every $a \in HxK$ there is an $R_l$ that contains $a$. So $\cup
        R_k = HxK$. Also $\cup R_k$ cannot exceed $HxK$ because each $R_k$
        contains only elements from $HxK$.\\

  \item Suppose $H$ and $K$ are finite. First we will prove that $|W| = |M|$.
        Let $f: W \to M$ by $f(w) = x^{-1}wx$. If $w \in W$ then we know
        $w \in H$ and $\exists k \in K$ where $w = xkx^{-1}$. Now, $f(w) =
        x^{-1}wx = x^{-1}xkx^{-1}x = k \in K$. So $f(w) \in K$ and also $f(w)
        \in x^{-1}Hx$. The function is a bijection with inverse given by
        $f^{-1}(m) = xmx^{-1}$. So $|W| = |M|$.

        We can show that $|HxK| = |H||K|/|W|$ relatively easily. If $h_1xk_1 =
        h_2xk_2 \in HxK$ then that means $h_1W = h_2W$. So this tells us the
        number of elements in $HxK$ is the product of the number of elements in
        $H$ and $K$, not counting elements congruent mod $W$. Thus:
        $$|HxK| = \frac{|H||K|}{|W|} = \frac{|H||K|}{|M|}.$$

\end{enumerate}

\noindent {\bf 2.8.}
\begin{enumerate}
  \item \begin{itemize}
        \item Let $x,y \in C_G(A)$, consider $xy^{-1}$. We know $ax = xa$ for
              all $a \in A$ and $ay = ya \to a = yay^{-1} \to y^{-1}a =
              ay^{-1}$. Therefore $axy^{-1}= (ax)y^{-1} = (xa)y^{-1} =
              x(ay^{-1}) = xy^{-1}a$ and $xy^{-1} \in C_G(A)$. So $C$ is a
              subgroup of $G$.

        \item Let $x,y \in N_G(A)$, consider $xy^{-1}$. We know $xAx^{-1} = A$
              and $yAy^{-1} = A \to A = y^{-1}Ay$. Therefore $xy^{-1}A(xy^{-1})
              ^{-1}=x(y^{-1}Ay)x^{-1} = xAx^{-1} = A$. We're done, $N_G(A)$ is a
              subgroup of G.

        \item Let $x \in C_G(A)$, so for all $a \in A$ $xa = ax$. This implies
        that $xA = Ax$. This means $xAx^{-1} = A$, so $x \in N_G(A)$.
        \end{itemize}

  \item Suppose that $A$ is a subgroup of $G$. Let $n \in N_G(A)$. We have
        $nAn^{-1} = A$ by definition of $N_G(A)$. This means $nA = An$ for all
        $n \in N_G(A)$ and thus $A \trianglelefteq N_G(A)$ and we're done.

  \item Consider $g \in G$. We know that for all $z \in Z(G)$, $gz = zg$. So
        $gZ(G) = Z(G)g$ and thus $Z(G) \trianglelefteq G$.

  \item Let $[G:H] = 2$ Consider $g \in G$ either $g \in G-H$ or $g \in H$, if
        $g \in H$ then clearly $gH = Hg$. If $g \in G-H$ then $gH = G-H$ because
        $H$ has index $2$ and there are only $2$ cosets. From the same logic we
        have $Hg = G-H$, so $gH = Hg$ and $H$ is normal.\\

        For an example, consider $D_6$. $[D_6:\langle s \rangle] = 3$ but $sr
        \neq rs$ so $\langle s \rangle$ is not a normal subgroup of $D_6$.

  \item If $n$ is even, $Z(D_{2n}) = \langle r^{n/2} \rangle$. If $n$ is odd
        $Z(D_{2n}) = e$. We know the rotations will always commute, and that
        $sr = r^{-1}s$. This means only elements with $r = r^{-1}$ will be in
        $Z(D_{2n})$, that is $r^2 = 1$. So the center of the odd groups is just
        the identity, and for the even ones it is the identity and $r^{n/2}$.

  \item Let $N$ be a subgroup of $Q_8$. We know from the structure of $Q_8$ that
        for every $q \in Q$ and every $n \in N$ either $qn = nq$ or $qn = -nq$.
        If $N$ is just the group containing $1$ then we're done. If $N$ contains
        an element $n \neq 1$, then $N$ must contain $-n$ because $i*(-i) =
        j*(-j) = k*(-k) = 1$ and $-(-1) = 1$. So $N = -N$ as long as
        $N \neq \{ 1\}$. Then we have $qN = Nq$ for all $q \in Q$. Done.

\end{enumerate}

\noindent {\bf 2.9 \#1.}
Let $G$ have prime order. We know that if $H$ is a subgroup of $G$ then $|H|$
divides $|G|$. However, since $|G|$ is prime, we have $|H| = 1$ or $|H| = |G|$.
Thus $H = G$ or $H = 1$ and $G$ has no nontrivial subgroups. This proves $G$ has
no non-trivial subgroups.

Also, $\forall g\in G$ we know $g^|G| = 1$, we also know that the order of each
element must divide evenly into $|G|$, therefore every element has order either
$|G|$ or $|1|$, we conclude there is at least one generator for $G$, so $G$ is
cyclic.

\noindent {\bf 2.9 \#2.}

Let $G$ be a group and $H$ be a subgroup of $Z(G)$, also suppose $G/H$ is
cyclic. Let $G/H = \langle g \rangle$ and let $a,b \in G$. We can write $a =
g^nh_1$ and $b = g^mh_2$. Then we use group properties:
\begin{align*}
 ab                            &= g^nh_1g^mh_2 \\
\textrm{Since $h_1 \in Z(G) $} &= g^ng^mh_1h_2 \\
                               &= g^{n+m}h_1h_2\\
                               &= g^mg^nh_2h_1 \\
\textrm{Since $h_2 \in Z(G) $} &= g^mh_2g^nh_1 \\
                               &= ba
\end{align*}
So we have $G$ is Abelian.\\

\noindent {\bf 2.10.}

\begin{figure}[h]
  \centering
\begin{tikzpicture}[node distance=2cm]
\title{Subgroups of $\mathbb{D}_{6}$}
\node(D3)                           {$D_6$};
\node(R)       [below left of=D3] {$\langle r \rangle$};
\node(S)       [below of=D3]      {$\langle s \rangle$};
\node(RS)      [right of=S]       {$\langle rs \rangle$};
\node(R2S)     [right of=RS]      {$\langle r^2s \rangle$};
\node(1)       [below of=S]       {$\langle 1 \rangle$};

\draw(D3)      --  (R);
\draw(D3)      --  (S);
\draw(D3)      --  (RS);
\draw(D3)      --  (R2S);
\draw(R)       --  (1);
\draw(S)       --  (1);
\draw(RS)      --  (1);
\draw(R2S)     --  (1);


\end{tikzpicture}
\caption{Subgroups of $\mathbb{D}_{6}$}
\end{figure}

\begin{figure}[h]
  \centering
\begin{tikzpicture}[node distance=2cm]
\title{Subgroups of $\mathbb{D}_{8}$}
\node(D8)                           {$D_8$};
\node(R)       [below of=D8]        {$\langle r       \rangle$};
\node(R2S)     [left of=R]   {$\langle s, r^2  \rangle$};
\node(RS)      [right of=R]  {$\langle rs, r^2 \rangle$};
\node(R2)      [below of=R]         {$\langle r^2     \rangle$};

\node(SR2)     [below of=R2S]       {$\langle r^2s    \rangle$};
\node(S)       [left of=SR2]  {$\langle s       \rangle$};
\node(SR)      [below of=RS]        {$\langle rs      \rangle$};
\node(SR3)     [right of=SR]  {$\langle r^3s    \rangle$};
\node(1)       [below of=R2]        {$\langle 1       \rangle$};

\draw(D8)      --  (R2S);
\draw(D8)      --  (R);
\draw(D8)      --  (RS);
\draw(R2S)     --  (S);
\draw(R2S)     --  (R2);
\draw(R2S)     --  (SR2);
\draw(R)       --  (R2);
\draw(RS)      --  (SR);
\draw(RS)      --  (R2);
\draw(RS)      --  (SR3);
\draw(R2)       --  (1);
\draw(S)       --  (1);
\draw(SR)       --  (1);
\draw(SR2)      --  (1);
\draw(SR3)     --  (1);

\end{tikzpicture}
\caption{Subgroups of $\mathbb{D}_{8}$}
\end{figure}


\begin{figure}[h]
  \centering
\begin{tikzpicture}[node distance=2cm]
\title{Subgroups of $\mathbb{Q}_{8}$}
\node(Q8)                         {$Q_8$};
\node(J)      [below of=Q8]       {$\langle j \rangle$};
\node(I)      [left of=J]         {$\langle i \rangle$};
\node(K)      [right of=J]        {$\langle k \rangle$};
\node(C)      [below of=J]        {$\langle -1 \rangle$};
\node(1)      [below of=C]        {$\langle 1 \rangle$};

\draw(Q8)      --  (I);
\draw(Q8)      --  (J);
\draw(Q8)      --  (K);
\draw(I)       --  (C);
\draw(J)       --  (C);
\draw(K)       --  (C);
\draw(C)       --  (1);



\end{tikzpicture}
\caption{Subgroups of $\mathbb{Q}_{8}$}
\end{figure}
\end{document}
