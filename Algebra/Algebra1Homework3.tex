\documentclass[12pt]{report}
\usepackage[margin=0.5in]{geometry}
\usepackage{amssymb,textcomp}
\title{\textbf{Algebra I: Homework 2}}
\author{Lee Fisher}
\date{}
\begin{document}

\textbf{Algebra 1 Homework 3}\\
\indent \textbf{Lee Fisher}\\
\indent \textbf{2017-09-09}

\begin{itemize}

\item Page 40 \#2. Consider $\phi: G \to H$ an isomorphism. Let $x \in G$ with
      $|x| = n$. This means $x^n=1_G$. Therefore $\phi(x^n) = \phi(1_G)$, and
      since $\phi$ is an isomorphism we have $\phi(x)^n = 1_H$. So the order of
      $\phi(x)$ is at most $n$.\\

      To prove the order is equal to $n$ suppose there is a $k<n$ for which
      $\phi(x)^k = 1_H$, this means $\phi(x^k) = 1_H$ and that $x^k =
      \phi^{-1}(1_H)$. Finally $x^k = 1$ which contradicts $|x|=n$. So
      $|\phi(x)| = |x|$.\\

      If $\phi$ is only a homomorphism we don't get this result. Take
      $\phi:(\mathbb{Z},+) \to (\mathbb{Z_2},+)$ by $\phi(x) = x \textrm{mod}2$.
      $\phi$ is a homomorphism but $|1| = \infty$ and $|\phi(1)| = 2$.\\

\item Page 40 \#3. Let $\phi: G \to H$ be an isomorphism. Consider $a,b \in G$.
      We have $\phi(a)\phi(b) = \phi(ab) = \phi(ba) = \phi(b)\phi(a)$. So
      $\textrm{Im}(\phi)$ is commutative. Because $\phi$ is a bijection we have
      $\textrm{Im}(\phi) = H$, so $H$ is abelian.\\

      For the other direction note that $\phi^{-1}$ is also an isomorphism, and
      therefore if $H$ is abelian then $G$ will be abelian by the same argument.\\

      More generally, if $\phi:G \to H$ is a homomorphism and $G$ is abelian
      then $H$ will be abelian provided $\textrm{Im}(\phi) = H$. In other words,
      so $\phi$ must be onto to ensure that if $G$ is abelian, then so is $H$.\\

\item Page 40 \#4. Consider the multiplicative groups $\mathbb{R}-{0}$ and
      $\mathbb{C}-{0}$. Consider, for sake of contradiction, an isomorpism
      $\phi: \mathbb{C}-{0} \to \mathbb{R}-{0}$. Now, $\phi(i) \in
      \mathbb{R}-{0}$. Say $\phi(i) = x$, then we have $\phi(i)^2 = x^2$, which
      means $x^2 = \phi(-1)$.\\

      We will now prove $\phi(-1) = -1$. We have $\phi(-1)^2 = \phi(1) = 1$.
      This means $\phi(-1)$ equals either $1$ or $-1$. If $\phi(-1) = 1$ then
      $\phi(-1) = \phi(1)$ which contradicts $\phi$ being one to one. So, since
      $\phi(-1) = -1$ we have $x^2 = -1$, where $x$ is real. This equation has
      no solutions over the real numbers, this contradicts $\phi$ being well
      defined. Thus the multiplicative groups $\mathbb{C} -{0}$ and
      $\mathbb{R} -{0}$ are not isomorphic.\\

\item Page 40 \#7. $D_8$ and $Q_8$ are not isomorphic. $D_8$ has 4 elements of
      order 2 (V,H,D,D') while $Q_8$ has only one element of order 2 (-1).\\

\item Page 40 \#17. Let G be a map and consider the map $\phi:G \to G$ by
      $\phi(g) = g^{-1}$.\\

      $\rightarrow$ Suppose $\phi$ is a homomorphism. Consider $a,b \in G$
      then we have $b^{-1}a^{-1} = (ab)^{-1} = \phi(ab) = \phi(a)\phi(b) =
      a^{-1}b^{-1}$. So $b^{-1}a^{-1} = a^{-1}b^{-1}$ for all $a,b \in G$. Thus
      $G$ is abelian.\\

      $\leftarrow$ Suppose $G$ is abelian. Consider $a,b \in G$. We have
      $\phi(ab) = (ab)^{-1} = b^{-1}a^{-1} = a^{-1}b^{-1} = \phi(a)\phi(b)$.
      Thus $\phi$ is a homomorphism.

\item Page 41 \#25.

\item Page 41 \#26.

\item Page 48 \#3

\item Page 48 \#10(a).

\item Page 60 \#1.

\item Page 60 \#3.

\item Page 60 \#12(a).
\end{itemize}

\end{document}
