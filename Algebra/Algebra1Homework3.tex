\documentclass[12pt]{report}
\usepackage[margin=0.5in]{geometry}
\usepackage{amssymb,textcomp,amsmath,tikz}
\title{\textbf{Algebra I: Homework 2}}
\author{Lee Fisher}
\date{}
\begin{document}

\textbf{Algebra 1 Homework 3}\\
\indent \textbf{Lee Fisher}\\
\indent \textbf{2017-09-09}

\begin{enumerate}

\item Page 40 \#2. Consider $\phi: G \to H$ an isomorphism. Let $x \in G$ with
      $|x| = n$. This means $x^n=1_G$. Therefore $\phi(x^n) = \phi(1_G)$, and
      since $\phi$ is an isomorphism we have $\phi(x)^n = 1_H$. So the order of
      $\phi(x)$ is at most $n$.

      To prove the order is equal to $n$ suppose there is a $k<n$ for which
      $\phi(x)^k = 1_H$, this means $\phi(x^k) = 1_H$ and that $x^k =
      \phi^{-1}(1_H)$. Finally $x^k = 1$ which contradicts $|x|=n$. So
      $|\phi(x)| = |x|$.

      If $\phi$ is only a homomorphism we don't get this result. Take
      $\phi:(\mathbb{Z},+) \to (\mathbb{Z_2},+)$ by $\phi(x) = x \textrm{mod}2$.
      $\phi$ is a homomorphism but $|1| = \infty$ and $|\phi(1)| = 2$.\\

\item Page 40 \#3. Let $\phi: G \to H$ be an isomorphism. Consider $a,b \in G$.
      We have $\phi(a)\phi(b) = \phi(ab) = \phi(ba) = \phi(b)\phi(a)$. So
      $\textrm{Im}(\phi)$ is commutative. Because $\phi$ is a bijection we have
      $\textrm{Im}(\phi) = H$, so $H$ is abelian.

      For the other direction note that $\phi^{-1}$ is also an isomorphism, and
      therefore if $H$ is abelian then $G$ will be abelian by the same argument.

      More generally, if $\phi:G \to H$ is a homomorphism and $G$ is abelian
      then $H$ will be abelian provided $\textrm{Im}(\phi) = H$. In other words,
      so $\phi$ must be onto to ensure that if $G$ is abelian, then so is $H$.\\

\item Page 40 \#4. Consider the multiplicative groups $\mathbb{R}-{0}$ and
      $\mathbb{C}-{0}$. Consider, for sake of contradiction, an isomorpism
      $\phi: \mathbb{C}-{0} \to \mathbb{R}-{0}$. Now, $\phi(i) \in
      \mathbb{R}-{0}$. Say $\phi(i) = x$, then we have $\phi(i)^2 = x^2$, which
      means $x^2 = \phi(-1)$.

      We will now prove $\phi(-1) = -1$. We have $\phi(-1)^2 = \phi(1) = 1$.
      This means $\phi(-1)$ equals either $1$ or $-1$. If $\phi(-1) = 1$ then
      $\phi(-1) = \phi(1)$ which contradicts $\phi$ being one to one. So, since
      $\phi(-1) = -1$ we have $x^2 = -1$, where $x$ is real. This equation has
      no solutions over the real numbers, this contradicts $\phi$ being well
      defined. Thus the multiplicative groups $\mathbb{C} -{0}$ and
      $\mathbb{R} -{0}$ are not isomorphic.\\

\item Page 40 \#7. $D_8$ and $Q_8$ are not isomorphic. $D_8$ has 4 elements of
      order 2 (V,H,D,D') while $Q_8$ has only one element of order 2 (-1).\\

\item Page 40 \#17. Let G be a map and consider the map $\phi:G \to G$ by
      $\phi(g) = g^{-1}$.\\

      $\rightarrow$ Suppose $\phi$ is a homomorphism. Consider $a,b \in G$
      then we have $b^{-1}a^{-1} = (ab)^{-1} = \phi(ab) = \phi(a)\phi(b) =
      a^{-1}b^{-1}$. So $b^{-1}a^{-1} = a^{-1}b^{-1}$ for all $a,b \in G$. Thus
      $G$ is abelian.\\

      $\leftarrow$ Suppose $G$ is abelian. Consider $a,b \in G$. We have
      $\phi(ab) = (ab)^{-1} = b^{-1}a^{-1} = a^{-1}b^{-1} = \phi(a)\phi(b)$.
      Thus $\phi$ is a homomorphism.

\item Page 41 \#25.
  \begin{itemize}
  \item Consider a vector in polar coordinates, and the product:
  \begin{align}
    \begin{bmatrix}
      \cos(\theta) & -\sin(\theta)\\
      \sin(\theta) & \cos(\theta)
    \end{bmatrix}
    \begin{bmatrix}
      r\cos(\phi)\\
      r\sin(\phi)
    \end{bmatrix}
    &= \begin{bmatrix}
      r\cos(\phi)\cos(\theta) - r\sin(\phi)\sin(\theta)\\
      r\cos(\phi)\sin(\theta) + r\sin(\phi)\cos(\theta)
    \end{bmatrix}\\
    &=\begin{bmatrix}
      r\cos(\phi+\theta)\\
      r\sin(\phi+\theta)
    \end{bmatrix}\\
  \end{align}
  Multiplication by this matrix will rotate a vector in $\mathbb{R}^2$ through
  an angle of $\theta$.

  \item We need to show that $\phi$ respects the group structure of $D_{2n}$ to
        prove that $\phi$ is a homomorphism. We need that $\phi(r)^n = \phi(s)^2
        = I$ and that $\phi(s)\phi(r) = \phi(r)^{-1}\phi(s)$.

        We know that $\theta = 2\pi/n$ and that $\phi(r)$ is a rotation matrix
        through and angle of $\theta$. If we multiply two rotation matrices
        together we will add their angles. So from this, we have $\phi(r)^n=I$.

        To show $\phi(s)^2 = I$ is a simple calculation:
        $$
        \begin{bmatrix}
          0 & 1\\
          1 & 0
        \end{bmatrix}
        \begin{bmatrix}
          0 & 1\\
          1 & 0
        \end{bmatrix}
        =
        \begin{bmatrix}
          1 & 0\\
          0 & 1
        \end{bmatrix}
        $$
        Done. For the last part we need $\phi(s)\phi(r) = \phi(r)^{-1}\phi(s)$.
        So, for the left hand side we have:
        $$
        \begin{bmatrix}
          0 & 1\\
          1 & 0
        \end{bmatrix}
        \begin{bmatrix}
          \cos(\theta) & -\sin(\theta)\\
          \sin(\theta) & \cos(\theta)
        \end{bmatrix}
        =
        \begin{bmatrix}
          \sin(\theta) & \cos(\theta)\\
          \cos(\theta) & -\sin(\theta)
        \end{bmatrix}
        $$

        And for the right hand side we have:
      \begin{align*}
        \begin{bmatrix}
          \cos(\theta) & -\sin(\theta)\\
          \sin(\theta) & \cos(\theta)
        \end{bmatrix}^{-1}
        \begin{bmatrix}
          0 & 1\\
          1 & 0
        \end{bmatrix}
        &=
        \begin{bmatrix}
           \cos(\theta) & \sin(\theta)\\
          -\sin(\theta) & \cos(\theta)
        \end{bmatrix}
        \begin{bmatrix}
          0 & 1\\
          1 & 0
        \end{bmatrix}\\
        &=
        \begin{bmatrix}
          \sin(\theta) & \cos(\theta)\\
          \cos(\theta) & -\sin(\theta)
        \end{bmatrix}
      \end{align*}

      So $\phi(r)$ and $\phi(s)$ satsify all the relations that generate
      $D_{2n}$. This means $\phi$ will be a homomorphism from $D_{2n}$ to
      $GL_2(\mathbb{R})$.

  \item In the previous part I showed that $\phi(r)$ and $\phi(s)$ satisfy all
  the relations that the generators for $D_{2n}$ satisfy. This means the
  image of $\phi$ will be isomorphic to $D_{2n}$ (with isomorphism $\phi$).
  So we know $\phi$ must be injective, otherwise $ |Im(\phi)| < |D_{2n}|$ which
  we know is impossible.

  \end{itemize}

\item Page 41 \#26. In the same way as the last problem we will show that
      $\phi(i)$ and $\phi(j)$ satisfy all the same relations as $i$ and $j$
      satisfy as generators of $Q_8$. That is: $\phi(i)^4 = I$,
      $\phi(i)^2 = \phi(j)^2$, and $\phi(j)^{-1}\phi(i)\phi(j)=\phi(i)^{-1}$.

      The first one is easy; since $\phi(i)$ is diagonal we have $\phi(i)^2 =
      -I$ and $(\phi(i)^2)^2 = \phi(i)^4 = I$. We get the second one almost as
      easily:
      $$
      \phi(j)^2=
      \begin{bmatrix}
        0 & -1\\
        1 & 0
      \end{bmatrix}
      \begin{bmatrix}
        0 & -1\\
        1 & 0
      \end{bmatrix}
      =
      \begin{bmatrix}
        -1 & 0\\
        0 & -1
      \end{bmatrix}
      =\phi(i)^2
      $$

      Then we have to prove the third relation:

      \begin{align*}
        \begin{bmatrix}
          0 & -1\\
          1 &  0
        \end{bmatrix}^{-1}
        \begin{bmatrix}
          \sqrt{-1} & 0\\
          0 & -\sqrt{-1}
        \end{bmatrix}
        \begin{bmatrix}
          0 & -1\\
          1 & 0
        \end{bmatrix}
        &=
        \begin{bmatrix}
           0 & 1\\
          -1 & 0
        \end{bmatrix}
        \begin{bmatrix}
          \sqrt{-1} & 0 \\
           0 & -\sqrt{-1}
        \end{bmatrix}
        \begin{bmatrix}
          0 & -1\\
          1 & 0
        \end{bmatrix}\\
        &=
        \begin{bmatrix}
          0 & -\sqrt{-1}\\
         -\sqrt{-1} & 0
        \end{bmatrix}
        \begin{bmatrix}
          0 & -1\\
          1 & 0
        \end{bmatrix}\\
        &=
        \begin{bmatrix}
          -\sqrt{-1} & 0 \\
           0 & \sqrt{-1}
        \end{bmatrix}
      \end{align*}

      This proves it. We know that $\phi(i)$ and $\phi(j)$ as elements of
      $GL_2(\mathbb{C})$ satisfy all the relations that generate $Q_8$. So just
      like before, $\phi$ will be an injective homomorphism, and the image of
      $\phi$ will be isomorphic to $Q_8$.

\item Page 48 \#3
      We'll do both parts by constructing the Cayley tables.
      \begin{table}[h]
      \centering
      \caption{The Other Subset}
      \begin{tabular}{c|cccc}
        $\circ$ &1      &$r^2$  &$s$    &$sr^2$  \\
        \hline
        1       &1      &$r^2$  &$s$    &$sr^2$  \\
        $r^2$   &$r^2$  &1      &$sr^2$ &$s$     \\
        $s$     &$s$    &$sr^2$ &1      &$r^2$   \\
        $sr^2$  &$sr^2$ &$s$    &$r^2$  &1
      \end{tabular}
      \end{table}

      \begin{table}[h]
      \centering
      \caption{One Subset}
      \begin{tabular}{c|cccc}
        $\circ$ &1      &$r^2$  &$sr$   &$sr^3$  \\
        \hline
        1       &1      &$r^2$  &$sr$   &$sr^3$  \\
        $r^2$   &$r^2$  &1      &$sr^3$ &$sr$    \\
        $sr$    &$sr$   &$sr^3$ &1      &$r^2$   \\
        $sr^3$  &$sr^3$ &$sr$   &$r^2$  &1
      \end{tabular}
      \end{table}
      Because their Cayley Tables are both tables of groups of order 4, these
      sets must be subgroups.

\item Page 48 \#10(a). Let $H$ and $K$ be subgroups of $G$. Consider
      $H \cap K$. To prove $H \cap K$ is a subgroup we will show it is closed
      under multiplication and inverses. Let $x \in H \cap K$, because $x \in H$
      and $H$ is a subgroup $x^{-1} \in H$; likewise $x^{-1} \in K$.
      Therefore $x^{-1} \in H \cap K$.

      Consider $x,y \in H \cap K$. Well, $x,y \in H$ so $xy \in H$, likewise
      $xy \in K$. So we have $xy \in H \cap K$. We conclude $H \cap K$ is a
      subgroup.

\item Page 60 \#1.
      Find all the subgroups of $Z_45 = \langle x \rangle$, giving a generator
      for each. $Z_45$ has subgroups of order 45, 15, 9, 5, 3, and 1. To
      generate these subgroups we can do it in this order: $|\langle 1 \rangle|
      = 45$, $|\langle 3 \rangle| = 15$, $|\langle 5 \rangle| = 9$, $|\langle 15
      \rangle| = 3$, $|\langle 0 \rangle| = 1$. The picture of subgroup
      containment looks like this.

\begin{figure}[h]
  \centering
\begin{tikzpicture}[node distance=2cm]
\title{Subgroups of $\mathbb{Z}_{45}$}
\node(Z1)                           {$\langle 1 \rangle$};
\node(Z5)       [below right of=Z1] {$\langle 5 \rangle$};
\node(Z3)       [below left of=Z1]  {$\langle 3 \rangle$};
\node(Z15)      [below left of=Z5]  {$\langle 15 \rangle$};
\node(Z9)       [below left of=Z3]  {$\langle 9 \rangle$};
\node(Z0)       [below left of=Z15] {$\langle 0 \rangle$};

\draw(Z1)      --  (Z5);
\draw(Z1)      --  (Z3);
\draw(Z5)      --  (Z15);
\draw(Z3)      --  (Z15);
\draw(Z3)      --  (Z9);
\draw(Z15)     --  (Z0);
\draw(Z9)      --  (Z0);


\end{tikzpicture}
\caption{Subgroups of $\mathbb{Z}_{45}$}
\end{figure}
\item Page 60 \#3. The generators of $\mathbb{Z}/48\mathbb{Z}$ will be numbers
      less than $48$ and relatively prime with $48$. These are 1, 5, 7, 11, 13,
      17, 19, 23, 25, 29, 31, 35, 37, 41, 43, and 47.

\item Page 60 \#12(a). $Z_2 \times Z_2$ is not cyclic. The group has order $4$
      but the orders of its elements are: $|(0,0)|=1$, $|(1,0)|=2$, $|(1,1)|=2$,
      and $|(0,1)|=2$. $Z_2 \times Z_2$ has no elements of order 4 so it is not
      cyclic.
\end{enumerate}

\end{document}
