\documentclass[12pt]{report}
\usepackage[margin=0.5in]{geometry}
\usepackage{amssymb,amsmath,textcomp}
\usepackage{parskip}
\providecommand{\abs}[1]{\lvert#1\rvert} \providecommand{\norm}[1]{\lVert#1\rVert}
\setlength{\parindent}{1cm} % Default is 15pt.
\linespread{1.5}
\title{\textbf{Algebra I: Homework 1}}
\author{Lee Fisher}
\date{}
\begin{document}

\noindent \textbf{Topology Homework 2}\\
\noindent \textbf{Lee Fisher}\\
\noindent \textbf{2017-09-12}


\noindent {\bf Problem 1:} Show that a space is simply connected if and only if
all paths having the same endpoints are fixed endpoint homotopic.

$\rightarrow$ Suppose $X$ is simply connected. Consider two points $x_0$ and
$x_1$, and two paths $\alpha$ and $\beta$ connecting the points. We say
$\bar{\beta}$ is $\beta(1-t)$. We can say that $[\alpha] \cong [\alpha] *
[\bar{\beta} * \beta]$ because the path $[\bar{\beta} * \beta]$ is homotopic to
a constant path at $x_1$. Then by associativity we have $[\alpha] \cong [\alpha
* \bar{\beta}] * [\beta]$. Next, because $[\alpha *\bar{\beta}]$ is a loop
based at $x_0$ we have $[\alpha] \cong [1] * [\beta]$. Finally we get $[\alpha]
\cong [\beta]$ because $[1]$ is the identity. Thus $\alpha$ and $\beta$ are
fixed endpoint homotopic.

$\leftarrow$ Suppose any two paths in $X$ are fixed end point homotopic.
Consider a loop that begins and ends at the point $x_0$ and a path that is
constant at $x_0$. These paths have the same endpoints so they are homotopic.
Thus any loop based at $x_0$ will be homotopic to the path that is constant at
$x_0$, and $X$ is by definition simply connected.\\

\noindent {\bf Problem 2:} We need to show that $(g \circ f)_* = g_* \circ f_*$.

This is not so hard. Let $[\gamma] \in \pi_1(X,x_0)$. Then we have $(g \circ
f)_* \circ [\gamma] = [g \circ f \circ \gamma]$ by definition. On the other
hand, lets look at $g_* \circ f_* \circ [\gamma]$, this is $g_* (f_*([\gamma]))$
which is $g_*([f \circ \gamma])$, and finally $[g \circ f \circ \gamma]$. So the
maps are equal.\\

\noindent {\bf Problem 3:} Let $p:E \to B$ be a covering map with $p(e_0) =
b_0$. Let $F:[0,1]\times [0,1] \to B$ be continuous with $F(0,0) = b_0$. We want
to lift the entire function.

From the uniqueness of path lifting we can lift the edges of the domain, the
paths $F(t,0)$ and $F(0,s)$, to unique paths in $E$. Now we procede by cutting
the domain of $F$ into small closed cells. We start with two lists $0=t_1 < t_2
\dots < t_n = 1$ and $0 = s_1 < s_2 \dots < s_m =1$, where for each $I_a \times
J_b = [t_a,t_{a+1}] \times [s_b,s_{b+1}]$ we have an open set $U_{a,b}$
containing $F(I_a \times J_b)$ where $p^{-1}(U_{a,b})$ is a disjoint union of
open sets in $E$.

Next we will lift one square at a time. We'll first lift all of the squares in
row one $I_a \times J_1$, and then procede to lift the next row, squares like
$I_a \times J_2$, and so on. Now let's lift the first square. We know that the
image of the bottom and left edges of this square lift to unique paths in $E$.
We also know that the image of the entire square will lift to a collection of
disjoint subsets of $E$. Since the unique paths $\tilde{F}(t,0)$ and
$\tilde{F}(0,s)$ agree on the point $(0,0)$, there will be a unique subset of
$E$ for which $\tilde{F}|_{I_1 \times J_1}$ will match $\tilde{F}$ on the
boundaries. In this way we extend the lift to the first square.

As we procede to all other squares we can see that each subsequent square will
share two adjacent sides with either some previous squares or the left or bottom
edges of the domain. In this way we can lift the entire function on $[0,1]
\times [0,1]$ to a unique $\tilde{F}$.

For the last part we suppose $F$ is a path homotopy. This means $F(t,0)$ and
$F(t,1)$ are fixed. When we specify that $\tilde{F}(0,0) = e_0$ we make a unique
lift to an $\tilde{F}$. This one must also have $\tilde{F}(t,0)$ and
$\tilde{F}(t,1)$ fixed. Then by continuity we have $\tilde{F}(0,s)$ and
$\tilde{F}(1,s)$ are two homotopic paths in $E$.\\

\noindent {\bf Problem 4:} Let $p: E \to B$ be a covering map where $B$ is
connected and there is some point $b \in B$ for which $|p^{-1}(b)| = k$.

Consider two subsets of $B$, where $U = \{x \in B : \abs{p^{-1}(x)} = k\}$ and
$V = \{y\in B:\abs{p^{-1}(y)} \neq k \}$. We can say that because $p$ is a
covering map if $x \in U$ there will be some open set containing $x$ that is a
subset of $U$; therefore $U$ is open. As well if $y \in V$ then there will be an
open set containing $y$ that is a subset of $V$; so $V$ also is open.

Since $p$ is an onto function we have that $U \cap V = \emptyset$ and $U \cup V
= B$. If $V$ is nonempty this forms a separation of $B$. So since $B$ is
connected we have $V = \emptyset$ and $U = B$. We conclude, every point has $k$
preimages.\\

\noindent {\bf Problem 5:} Suppose $B$ is simply connected, $E$ is path
connected, and $p:E \to B$ is a covering map with $p(e_0) = b_0$.

We can define a map $\phi: \pi_1(B,b_0) \to p^{-1}(b_0)$; we say if $[f] \in
\pi_1(B,b_0)$ then $[f]$ lifts to a unique class of paths beginning at $e_0$,
$[\tilde{f}]$, we say $\phi([f]) = [\tilde{f}](1)$. The map $\phi$ sends every
loop in $\pi_1(B,b_0)$ to the endpoint of its lift in $E$.

Since $E$ is path connected, the map $\phi$ will be surjective. Every path in
$E$ connecting $e_0$ to another point, say $e_1$, in $p^{-1}(b_0)$ will be
projected to a loop in $B$ by $p$. Therefore there will be at least one loop
for which $\phi([f]) = e_1$.

We know that $|\pi_1(B,b_0)| \geq |p^{-1}(b_0)|$ since $\phi$ is onto. As well
$\pi_1(B,b_0)$ is trivial, because $B$ is simply connected. This means
$|p^{-1}(b_0)| = 1$. So, $p$ covers the point $b_0$ by exactly one point. Now
by the previous we know that $p$ is a $1$-fold covering, also known as an
injective map. Since covering maps are surjective and continuous by supposition
we have that $p$ is a homeomorphism.\\

\noindent {\bf Problem 6:} Let $h: (X,x_0) \to (Y,y_0)$ be an inessential map.
We want to show that $h_*$ is trivial.

We know that there is a homotopy $\eta$ that deforms $h$ to a constant map on
$Y$. Consider the induced map $h_*: \pi_1(X,x_0) \to \pi_1(Y,y_0)$. If $[x] \in
\pi_1(X,x_0)$ we can consider $h_* \circ[x] = [h \circ x]$ and since $\eta$ is a
homotopy we have $\eta \circ [h \circ x] \cong [h \circ x]$, but also $\eta
\circ [h \circ x] \cong [(\eta \circ h) \circ x] \cong [1_{y_0} \circ x] \cong
[1]$. This tells us that $h_*([x]) = 1$. So the map induced by an inessential
map is trivial.


\end{document}
