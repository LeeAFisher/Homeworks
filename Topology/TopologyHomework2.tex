\documentclass[12pt]{report}
\usepackage[margin=0.5in]{geometry}
\usepackage{amssymb,amsmath,textcomp}
\providecommand{\abs}[1]{\lvert#1\rvert} \providecommand{\norm}[1]{\lVert#1\rVert}

\title{\textbf{Algebra I: Homework 1}}
\author{Lee Fisher}
\date{}
\begin{document}

\textbf{Topology Homework 2}\\
\indent \textbf{Lee Fisher}\\
\indent \textbf{2017-09-12}

\begin{enumerate}
\item Show that a space is simply connected if and only if all paths having the
same endpoints are fixed endpoint homotopic.

$\rightarrow$ Suppose $X$ is simply connected. Consider two points $x_0$ and 
$x_1$, and two paths $\alpha$ and $\beta$ connecting the points. We say 
$\bar{\beta}$ is $\beta(1-t)$. We can say that $[\alpha] \cong [\alpha] * 
[\bar{\beta} * \beta]$ because the path $[\bar{\beta} * \beta]$ is homotopic to
a constant path at $x_1$. Then by associativity we have $[\alpha] \cong [\alpha 
* \bar{\beta}] * [\beta]$. Next, because $[\alpha *\bar{\beta}]$ is a loop 
based at $x_0$ we have $[\alpha] \cong [1] * [\beta]$. Finally we get $[\alpha]
\cong [\beta]$ because $[1]$ is the identity. Thus $\alpha$ and $\beta$ are 
fixed endpoint homotopic.


$\leftarrow$ Suppose any two paths in $X$ are fixed end point homotopic. 
Consider a loop that begins and ends at the point $x_0$ and a path that is 
constant at $x_0$. These paths have the same endpoints so they are homotopic.
Thus any loop based at $x_0$ will be homotopic to the path that is constant at 
$x_0$, and $X$ is by definition simply connected. 

\item We need to show that $(g \circ f)_* = g_* \circ f_*$. This is not so 
hard. Let $[\gamma] \in \pi_1(X,x_0)$. Then we have $(g \circ f)_* \circ 
[\gamma] = [g \circ f \circ \gamma]$ by definition. On the other hand, lets 
look at $g_* \circ f_* \circ [\gamma]$, this is $g_* ( f_*([\gamma]))$ which is
$g_*([f \circ \gamma])$, and finally $[g \circ f \circ \gamma]$. So the maps are 
equal.


\item 


\item Let $p: E \to B$ be a covering map where $B$ is connected and there is 
some point $b \in B$ for which $|p^{-1}(b)| = k$. Consider two subsets of $B$
where $U = \{x \in B : \abs{p^{-1}(x)} = k \}$ and $V = \{y \in B : \abs{p^{-1}(y)} 
\neq k \}$. We can say that if $x \in U$, because $p$ is a covering map, there 
will be some open set containing $x$ that is in $U$; therefore $U$ is open. As 
well if $y \in V$ then there will be an open set containing $y$ that is in $V$;
so $V$ also is open. Since $p$ is an onto function we have that $U \cap V = 
\emptyset$ and $U \cup V = B$. If $V$ is nonempty this forms a separation of 
$B$. So since $B$ is connected we have $V = \emptyset$ and $U = B$.  

\item 
\item 
\end{enumerate}

\end{document}
