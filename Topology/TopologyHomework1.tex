\documentclass[12pt]{report}
\usepackage[margin=0.5in]{geometry}
\usepackage{amssymb,amsmath,textcomp}
\title{\textbf{Algebra I: Homework 1}}
\author{Lee Fisher}
\date{}
\begin{document}

\textbf{Topology Homework 1}\\
\indent \textbf{Lee Fisher}\\
\indent \textbf{2017-08-30}

\begin{enumerate}
\item Construct several examples of homotopic and non-homotopic maps. No proofs required.
Here are some examples of homotopic maps:
\begin{itemize}
  \item The straight lines $\vec{r}_1(t) = (2t+1,t-3)$ and $\vec{r}_2(t) = (4t+5,-6t+2)$ will be homotopic in $\mathbb{R}^2$.
  \item Any two paths on a Torus will be homotopic because the Torus is path connected.
  \item Pick two paths in the upper half plane of $\mathbb{R}^2$ without the $x$-axis.
\end{itemize}

Here are some examples of non-homotopic maps:
\begin{itemize}
  \item In $\mathbb{R}^2$ without the $x$-axis, take one path in the upper half plane and another path in the
        lower half plane. These paths won't be homotopic in this space.
  \item Our space is the image of the sine curve $f(x) = sin(1/x)$, for all $x >0$ union the closed interval of the
        y-axis, $[-1,1]$. Take one path to be the closed interval $[-1,1]$ and the other to be a connected piece of the
        sine curve. This space isn't path connected, so the two paths won't be homotopic.
  \item Take the space to be $R^3$ without the xy-plane. Pick one path to have only positive $z$-coordinates and the
        other to have only negative z-coordinates. These paths won't be homotopic.
\end{itemize}

\item To prove that fixed endpoint homotopy is an equivalence relation we need to prove three things.
\begin{itemize}
  \item Reflexivity. Consider a path $f$, a homotopy from $f$ to $f$ would be the constant map $\phi(x,t) = f(x)$.
  \item Symmetry. Suppose $\phi(x,t)$ is a homotopy from $f_1$ to $f_2$. Then $\phi(x,1-t)$ will be a homotopy from
        $f_2$ to $f_1$.
  \item Transitivity. Suppose $\phi_1(x,t)$ is a homotopy from $f_1$ to $f_2$ and $\phi_2(x,t)$ is a homotopy from $f_2$
        to $f_3$. Then we a define a new homotopy:
        \[ \phi_3(x,t) = \begin{cases}
        \phi_1(x,2t) & t \in [0,1/2] \\
        \phi_2(x,2t-1) & t \in (1/2,1]
        \end{cases} \]
        $\phi_3(x,t)$ satisfies $\phi_3(x,0) = f_1$ and $\phi_3(x,1) = f_3$. As well $\phi_3$ is continuous at the point
        $t = 1/2$ because $\phi_1(x,2*1/2) = \phi_2(x,2*1/2-1) = f_2$. This means homotopy is transitive.
\end{itemize}

\item Here are some examplse of fixed endpoint homotopic maps:
\begin{itemize}
  \item Two paths in $\mathbb{R}^2$ with the same endpoints will be fixed endpoint homotopic.
  \item Two paths in $\mathbb{R}^3$ with the same endpoints will be fixed endpoint homotopic.
        (Sorry... I'm having difficulty thinking of interesting ways for things to be homotopic.)
  \item Two paths in the upper half plane of $\mathbb{R}^2$ without the $x$-axis will be fixed endpoint homotopic.
\end{itemize}

Here are some examples of non fixed-endpoint homotopic maps:
\begin{itemize}
  \item Take the paths $(\cos(\pi t),\sin(\pi t)) $ and $(\cos(\pi t), -\sin(\pi t))$ in $\mathbb{R}^2$ without the
        origin. These will not be fixed endpoint homotopic.
  \item These paths, $(\cos(\pi t),\sin(\pi t),0)$ and $(\cos(\pi t), -\sin(\pi t),0)$ in $\mathbb{R}^3$ without
        the $z$-axis.
  \item Two halves of a circle in $S^1$ will also work (or not work I guess).
\end{itemize}


\item A space $X$ is contractible if the identity map $\textrm{Id}_X : X \to X$ is homotopic to a constant map.
\begin{itemize}
  \item Convex open subsets of $\mathbb{R}^n$ are contractible. A convex subset, $X$ is by definition one in which the
  straight line between any two points in $X$ inside of $X$. Specifically, there exists a point ${\bf x}_0$ for which
  every point in the set $X$ is connected to that point by a straight line. If ${\bf x} \in X$ then define the homotopy
  $\phi$, by $\phi({\bf x},t) = t({\bf x}_0 - {\bf x}) + {\bf x}$. This is a continuous map and contracts every point in
  the convex set $X$ to a single point.

  \item Consider a contractible space $X$, and two points $a$ and $b$ in $X$. We know there is a map $\phi(x,t)$ which
  contracts every point in $X$ to a single point $x_0$. To make a path $\gamma(t)$ from $a$ to $b$ take:

  \[ \gamma(t) = \begin{cases}
  \phi(a,2t) & t \in [0,1/2] \\
  \phi(b,-2t+2) & t \in (1/2,1]
  \end{cases} \]

  This is a continuous path from $a$ to $b$ because $\gamma(\frac{1}{2}) = x_0$. This proves contractible spaces are
  path connected.

  \item Suppose $Y$ is a contractible space. Consider two maps $f_1: X \to Y$ and $f_2: X \to Y$. Since $Y$ is
  contractible, there is a contraction map, $\phi$ which sends every point to element of $Y$, $x_0$. We can define a
  homeomorphism, say $\eta$ from $f_1$ to $f_2$ in a similar way as before.
  \[ \eta(x,t) = \begin{cases}
  \phi(f_1(x),2t) & t \in [0,1/2] \\
  \phi(f_2(x),-2t+2) & t \in (1/2,1]
  \end{cases} \]

  This is a continuous map because both cases match at the point $t = \frac{1}{2}$, and $\eta(x,0) = f_1$ and
  $\eta(x,1) = f_2$. Thus, any two maps from $X \to Y$ will be homeomorphic.


  \item Suppose $X$ is contractible and $Y$ is path-connected. Consider two maps $f_1:X \to Y$ and $f_2:X \to Y$.
  To make a homeomorphism from $f_1$ to $f_2$ we need a contraction, $\phi$ of $X$ to $x_0$ and a path $p$ from
  $f_1(x_0)$ to $f_2(x_0)$. Then we define a homeomorphism:

  \[ \eta(x,t) = \begin{cases}
  f_1(\phi(x,3t)) & t \in [0,1/3] \\
  p(3t-1) & t \in (1/3, 2/3]\\
  f_2(\phi(x,-3t+3)) & t \in (2/3, 1)
  \end{cases} \]

  This is a homeomorphism from $f_1$ to $f_2$. $\eta$ first contracts the domain of the map $f_1$ to a point, then
  $\eta$ moves the image $f_1(x_0)$ to the point $f_2(x_0)$ along a path in $Y$, and then lastly decontracts the domain
  back to $X$ from $x_0$. $\eta(x,0) = f_1(x)$ and $\eta(x,1) = f_2(x)$. $\eta$ is made of continuous pieces that agree
  at the transition points $t=1/3$ and $t = 2/3$.\\

  If $Y$ is not path connected, then the maps $f_1$ and $f_2$ could fail to be homeomorphic if they don't satisfy
  $f_1(x_0) = f_2(x_0)$, for some contraction point $x_0$.
\end{itemize}

\item The fundamental group $\pi_1(X,x_0)$ is a group.
\begin{itemize}
  \item Closure: Consider $\phi_1$ and $\phi_2$ equivalence class representatives of loops in $\pi_1(X,x_0)$.
        The composition \[ \phi_1 * \phi_2 = \begin{cases}
        \phi_1(2t) & t \in [0,1/2] \\
        \phi_2(2t-1) & t \in (1/2,1]
        \end{cases} \]
        Is a continuous map that takes on the value $x_0$ at the points $t = 0$ and $t = 1$, and thus is a loop and is
        an element of $\pi_1(X,x_0)$.

  \item Associativity: Consider three maps $\phi_1$, $\phi_2$, and $\phi_3$. Now compare the compositions
  $\phi_1 * (\phi_2 * \phi_3)$ and $(\phi_1 * \phi_2) * \phi_3$. By definition:
  \[ \phi_1 * (\phi_2 * \phi_3) = \begin{cases}
  \phi_1(2t) & t \in [0,1/2] \\
  \phi_2(4t-2) & t \in (1/2,3/4]\\
  \phi_3(4t-3) & t \in (3/4,1]\\
  \end{cases} \]

  and

  \[ (\phi_1 * \phi_2) * \phi_3 = \begin{cases}
  \phi_1(4t) & t \in [0,1/4] \\
  \phi_2(4t-1) & t \in (1/4,1/2]\\
  \phi_3(2t-1) & t \in (1/2,1]\\
  \end{cases} \]

  These two loops are homeomorphic. The following homeomorphism was constructed from pages of algebra and guesswork.
  It is not pretty but it works:

  \[ \eta(t,s) = \begin{cases}
  \phi_1\left(2t(1+s)\right)                       & t \in \left[0,\frac{1}{2s+2}\right] \\
  \phi_2\left(\frac{(s+2)(t(2s+2)-1)}{2s+1}\right) & t \in \left(\frac{1}{2s+2},\frac{3}{2s+4}\right]\\
  \phi_3\left(\frac{t(2s+4)-3}{2s+1}\right)        & t \in \left(\frac{3}{2s+4} ,1\right]
  \end{cases} \]

  It's easy enough to verify that $\eta(t,0) = \phi_1 * (\phi_2 * \phi_3)$ and $\eta(t,1) = (\phi_1 * \phi_2) * \phi_3$.
  As well, it's not difficult to see that $\eta(0,s) = \phi_1(0) = x_0$ and $\eta(1,s) = \phi_3(1) = x_0$. To show that
  the functions match at the other endpoints is more annoying, but... here we go!

  \begin{align*}
  \phi_1\left(2\left(\frac{1}{2s+2}\right)(1+s)\right)            &= \phi_1(1)\\
                                                                  &= x_0\\
  \phi_2\left(\frac{(s+2)((\frac{1}{2s+2})(2s+2)-1)}{2s+1}\right) &= \phi_2(0)\\
                                                                  &= x_0
  \end{align*}

  So far so good, here's the other one:

  \begin{align*}
  \phi_2\left(\frac{(s+2)((\frac{3}{2s+4})(2s+2)-1)}{2s+1}\right) &= \phi_2\left(\frac{(\frac{3(s+2)}{2s+4})(2s+2)-(s+2)}{2s+1}\right)\\
                                                                  &= \phi_2\left(\frac{\frac{3}{2}(2s+2)-(s+2)}{2s+1}\right)\\
                                                                  &= \phi_2\left(\frac{3(s+1)-(s+2)}{2s+1}\right)\\
                                                                  &= \phi_2\left(\frac{3s+3-s-2}{2s+1}\right)\\
                                                                  &= \phi_2\left(\frac{2s+1}{2s+1}\right)\\
                                                                  &= \phi_2(1)\\
                                                                  &= x_0\\
  \phi_3\left(\frac{\frac{3}{2s+4}(2s+4)-3}{2s+1}\right)          &= \phi_3\left(\frac{3-3}{2s+1}\right)\\
                                                                  &= \phi_3(0)\\
                                                                  &= x_0.
  \end{align*}
  Okay, so the functions used for the construction of this map are all continuous when $s,t \in [0,1]$, so this will be
  a homeomorphism from $\phi_1 * (\phi_2 * \phi_3)$ and $(\phi_1 * \phi_2) * \phi_3$. Therefore, we arrive at the
  conclusion of associativity.

  \item Identity Element: Anything in the homotopy class of the map constant at $x_0$ will be an indentity element of
  this group. If $\phi$ is a loop in $X$ then $x_0 * \phi$

  \[ \textrm{Id}_{x_0} * \phi = \begin{cases}
  x_0 & t \in [0,1/2] \\
  \phi(2t-1) & t \in (1/2,1]
  \end{cases} \]

  This is homeomorphic to $\phi$ with homeomorphism

  \[ \eta(t,s) = \begin{cases}
  x_0 & t \in [0,1/2(1-s)] \\
  \phi(t\frac{2}{s+1}+\frac{s-1}{s+1}) & t \in (1/2(1-s),1]
  \end{cases} \]

  This will satisfy $\eta(t,0) = x_0 * \phi$, $\eta(t,1) = \phi$, and be continuous for all $s,t\in [0,1]$.

  Likewise $\phi * x_0$ will be homeomorphic to $\phi$ with homeomorphism:

  \[ \eta(t,s) = \begin{cases}
  \phi(\frac{2t}{1+s}) & t \in [0,1/2(1+s)] \\
  x_0 & t \in (1/2(1+s),1]
  \end{cases} \]

  This will also be continuous and satisfy the similar corresponding constraints. This proves the identity element
  exists in the fundamental group.

  \item Inverse Elements: Consider a path $\phi(t)$ in $\pi_1(X,x_0)$. We denote $\phi^{-1}(t)$ as $\phi(1-t)$.
  To check that this is homeomorphic to the constant map we can compute $\phi*\phi^{-1}$

  \[ \phi * \phi^{-1} = \begin{cases}
  \phi(2t) & t \in [0,1/2] \\
  \phi(2-2t) & t \in (1/2,1]
  \end{cases} \]

  This is homeomorphic to the identity map with homeomorphism:

  \[ \eta(t,s) = \begin{cases}
  \phi(2t(1-s)) & t \in [0,1/2] \\
  \phi((2-2t)(1-s)) & t \in (1/2,1]
  \end{cases} \]

  This will satisfy $\eta(t,0) = \phi * \phi^{-1}$, $\eta(t,1) = x_0$, and be continuous for all $s,t\in [0,1]$.
  Done. $\pi_1(X,x_0)$ is a group.
\end{itemize}

\item Consider two groups $\pi_1(X,x_0)$ and $\pi_1(X,x_1)$ where $x_0$ and $x_1$ lie in the same path connected
      component. We will show that these two groups are isomorphic. Consider a path $p$ that goes from $x_0$ to $x_1$,
      and a path $p^{-1}$, the same path backwards, that goes from $x_1$ to $x_0$. Let $\phi_0$ be a loop based at
      $x_0$, it can be associated with a loop based at $x_1$ by the function, $f(\phi_0) = p^{-1}*\phi_0*p$. This
      function is an isomorphism because it is continuous and invertible. The inverse is given by: $f^{-1}(\phi_1) =
      p*\phi_1*p^{-1}$. Then to show that it works $f^{-1}(f(\phi_0)) = p*p^{-1}*\phi_0*p*p^{-1}$, from associativity
      and the fact that $p*p^{-1} = x_0 * \phi_0 * x_0$, and then again from associativity and $x_0$ being the identity
      element, we arrive at $x_0 * \phi_0 * x_0 = \phi_0$. Likewise
      $f(f^{-1}(\phi_0)) = p^{-1}*p*\phi_0*p^{-1}*p = x_0 * \phi_0 * x_0 = \phi_0$. Now we're done, we've showed that
      $f$ is invertible and continuous, and thus an isomorphism from $\pi_1(X,x_0)$ to $\pi_1(X,x_1)$.
\end{enumerate}

\end{document}
